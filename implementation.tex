\section{Implémentassions}

%il faut ajouter references
Les ACLs sont fréquemment implémentent comme extensions du noyaux, ça veut dire modules dans le LINUX contexte. L'objectif de ce cession c'est explique superficiellement les questions concernant les implémentions de les ACL. Le discussion doit lancer la base pour les évaluations de performance et les problèmes dans les sessions prochaines.

"Les ACLs sont morceaux d'information de taille variable qui sont associe avec les objet du système de fichier"\cite{aclsuse}. Plusieurs implémentations de cette modelé sont possible. Par exemple, avec Solaris dans le système de fichier UFS\cite{acl_permission} chaque \emph{inode} peut être avoir une ACL. Si il en a, il doit avoir l'information \emph{i\_shadow}, une pointeur pour une \emph{shadow inode}. Les \emph{shadow inode} sont comment fichiers régulières d'utilisateurs. Diffèrent fichier avec les mêmes ACL peut avoir pointeurs pour le même \emph{shadow inodes}. Les information des ACL sont garde dans les bloque de donné de chaque \emph{shadow inodes}.

La capacité de associer morceaux d'information avec fichiers est utilisé pour plusieurs fonctions du système de exploitation, alors, en la plupart de ces systèmes \emph{UNIX-like} (Linux comprendre) on trouve les Attributs Étendu (\emph{Extended Attributes (EAs)}). Les ACL sont implémentent avec ce mécanisme.

Le linux page de manuel\cite{aclsuse} \emph{attr(5)} contienne une explication précise sur les EAs dans linux, au notre but, suffi dire que comme les variables des processus, les EAs sont pairs (nom, valeur) associe de manier persistant avec les objet du système de fichier et que les appel de système linux, dans le espace de utilisateur, sont employé pour opérer sur les information de ces pairs dans le espace de adresse du noyaux. Aussi pour l'implémentation de cette infrastructure dans les système FreeBSD il faut voir le article de Robert Watson\cite{trust}. Cette article content aussi une comparaison de plusieurs implémentation de ces système. 

Dans le monde linux, aouter le supporte aux ACL avec une version limité des EA offre plusieurs avantages: Facilité de implémentation, opération atomique et interface \emph{stateless} que laisse aucun surcharge à cause de les \emph{file handlers}. On verrais après dans le cession de performance, que l'efficience est assez importante pour être oublier quand on parle de les donnés fréquentent accès comme les ACL.   

\subsection{Les EAs e les système de fichiers}
 
% At the file system level, the obvious and straight-forward approach to implement EAs is to create an additional directory for each file system object that has EAs and to create one file for each extended attribute that has the attribute's name and contains the attribute's value. Because on most file systems allocating an additional directory plus one or more files requires several disk blocks, such a simple implementation would consume a lot of space, and it would not perform very well because of the time needed to access all these disk blocks. Therefore, most file systems use different mechanisms for storing EAs.
