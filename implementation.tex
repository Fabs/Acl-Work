\section{Implementation}

% Extended Attributes
% 
% In this section we begin detailing the implementation of ACLs in Linux.
% 
% ACLs are pieces of information of variable length that are associated with file system objects. Dedicated strategies for storing ACLs on file systems might be devised, as Solaris does on the UFS file system [13]. Each inode on a UFS file system has a field called i_shadow. If an inode has an ACL, this field points to a shadow inode. On the file system, shadow inodes are used like regular files. Each shadow inode stores an ACL in its data blocks. Multiple files with the same ACL may point to the same shadow inode.
% 
% Because other kernel and user space extensions in addition to ACLs benefit from being able to associate pieces of information with files, Linux and most other UNIX-like operating systems implement a more general mechanism called Extended Attributes (EAs). On these systems, ACLs are implemented as EAs.
% 
% Extended attributes are name and value pairs associated permanently with file system objects, similar to the environment variables of a process. The EA system calls used as the interface between user space and the kernel copy the attribute names and values between the user and kernel address spaces. The Linux attr(5) manual page contains a more complete description of EAs as found on Linux. A paper by Robert Watson discussing supporting infrastructure for security extensions in FreeBSD contains a comparison of different EA implementations on different systems [25].
% 
% Other operating systems, such as Sun Solaris, Apple MacOS, and Microsoft Windows, allow multiple streams (or forks) of information to be associated with a single file. These streams support the usual file semantics. After obtaining a handle on the stream, it is possible to access the streams' contents using ordinary file operations like read and write. Confusingly, on Solaris these streams are called extended attributes as well. The EAs on Linux and several other UNIX-like operating systems have nothing to do with these streams. The more limited EA interface offers several advantages. They are easier to implement, EA operations are inherently atomic, and the stateless interface does not suffer from overheads caused by obtaining and releasing file handles. Efficiency is important for frequently accessed objects like ACLs.
% 
% At the file system level, the obvious and straight-forward approach to implement EAs is to create an additional directory for each file system object that has EAs and to create one file for each extended attribute that has the attribute's name and contains the attribute's value. Because on most file systems allocating an additional directory plus one or more files requires several disk blocks, such a simple implementation would consume a lot of space, and it would not perform very well because of the time needed to access all these disk blocks. Therefore, most file systems use different mechanisms for storing EAs.
