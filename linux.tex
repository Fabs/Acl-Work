\section{ACL en use}
%@@

Dans cette session on verrais les uses des ACL dans les système d'aujourd'hui. 

\subsection{ACL Kernel Patches}
Les ACL \emph{patches} ont été ajouter dans le noyaux Linux depuis November 2002. Cette \emph{patches} implémentent le POSIX 1003.1e brouillon 17 et elles ont été ajoute dans le version 2.5.46 du noyaux. Donc le support ACL et aussi présent dans le dernière version du noyaux aujourd'hui. Depuis 2004 le support aux ACL étions disponible pour les système de fichier Ext2, Ext3, IBM JFS, ReiserFS et SGI XFS. Les ACL sont supporte aussi pour le système NFS, par contre, il y a quelques problèmes de sécurité connu\cite{nfs_problem}. 

Aujourd'hui c'est assez simple pour ajouter le supporte aux ACL dans les distribution Linux comme Ubuntu ou Debian. On verrais les pas pour ajouter ce supporte après.

\subsection{Mac OSX}
Le système de exploitation Mac OSX (10.6.2 Snow Leopard dans le moment de écriture de ce article) a aussi les supporte aux ACL complètement intégrée dans l'interface de utilisateur (\ref{fig:img_mac-acl}). 

\begin{figure}[htbp]
	\centering
		\includegraphics[height=3in]{img/mac-acl.png}
	\caption{Mac OSX Snow Leopard ACL Interface}
	\label{fig:img_mac-acl}
\end{figure}


%Parler un peut plus de Mac.


\subsection*{Using ACL in Linux}


%References 
Les dernière version des distribution Debian ou Ubuntu, comme Ubuntu 9.10, dèjá vient avec le supporte aux ACL. Dans le Ubuntu 8.10 l'application Nautilus, qui est responsable pour la visualisation du système de fichier, contenait une interface pour les ACL, apparentement l'interface a été discontinue et le Nautilus du Ubuntu 9.10 n'en y a pas encore. Les pas pour ajouter le supporte dans le Ubuntu 9.10 sont:

\begin{verbatim}
1) Installer le paquet des acl. 
user@ubuntu:$ sudo apt-get install acl
 
2) Ajouter le option 'acl' au système de fichier correcte dans le /etc/fstab, comment:
UUID='gros sequence' /dev/hda6 /home ext3 rw,auto,acl 0 1

3) Remounter le systeme de fichier avec le nouvelle option
user@ubuntu:$ sudo mount /home -o remount

\end{verbatim}

\subsection*{Ajouter ACL aux fichiers}

On peut utiliser le commande 'ls -la' pour regarde les permission. Si une fichier contient information de sécurité avancée (comme \emph{access list}) on va voir le "character" '+', comment dans le sortie du command 'ls' ci-dessous (\ref{verb:ls}). Une fichier avec '@' était dire que le fichier a quelque EAs. 

\begin{center}
\label{verb:ls}
\begin{verbatim}
-rw-r--r--@ 1 fabsn  staff     378  8 Nov 15:29 Makefile
-rw-r--r--@ 1 fabsn  staff     618  8 Nov 15:59 README
-rw-r--r--@ 1 fabsn  staff      31  8 Nov 15:15 draft-header
-rw-r--r--@ 1 fabsn  staff      24  8 Nov 15:15 header
drwxr-xr-x@ 2 fabsn  staff     102  8 Nov 15:26 img
-rw-r--r--  1 fabsn  staff     972  8 Nov 15:57 rapport-draft.aux
-rw-r--r--  1 fabsn  staff   18129  8 Nov 15:57 rapport-draft.log
drwxrwxr-x+ 3 fabsn  staff	  1024  8 Nov 20:23 repertoire
\end{verbatim}
\end{center}

Pour voir les ACL on doit utilise le commande \emph{getfacl}. Regarde que les information sont ajoute d'accord avec les définition dans l'introduction sur les ACL dans la tabelle \ref{tab:entree}. 

\begin{verbatim}
fabsn@vadmin:/media/esisar$ getfacl repertoire/
# file: repertoire/
# owner: root
# group: root
user::r-x
user:daemon:rwx
user:bin:rwx
user:fabsn:rwx
user:nobody:rwx
group::r-x
group:admin:rwx
group:fabsn:rwx
mask::rwx
other::r-x	
\end{verbatim}

Aussi on a le commande \emph{setfacl} pour modifier, ou ajouter les permission ACL. Le commande dessous par exemple modifie (-m) les permission du utilisateur \emph{fabsn} pour le répertoire. 

\begin{verbatim}
setfacl -m u:fabsn:r-x repertoire
\end{verbatim}

\subsection*{Exemple ACL d'accès}

Voici une exemple d'utilisation trouvable dans le article d'Andreas Gruembacher\cite{aclsuse}.

On parte de la création de un répertoire avec l'application de umask de valeur 027 (octal). Ça veut dire que il va   désactiver l'écriture pour le groupe propriété et l'écriture, la lecture e l'exécution pour les autres.

\begin{verbatim}
$ umask 027 
$ mkdir dir 
$ ls -dl repertoire
	drwxr-x--- ... user group ... repertoire
\end{verbatim}

La lettre "d" montre que on l'objet "répertoire" c'est une répertoire suivi pour les octet de permission "rwxr-x---". Les réticences dans le command supprime les information avec aucune pertinence pour cette exemple. Cette permission basic on toujours une représentation dans les ACL, pour les afficher on faire:

\begin{verbatim}
$ getfacl repertoire
# file: repertoire 
# owner: user 
# group: goup
user::rwx
group::r-x
other::---
\end{verbatim}

Le trois première ligne après le commande (démarrer pour le \#) nous donne les information sur les propriétaire (groupe e utilisateur) et le nom du fichier. Après chaque ligne vient la liste des ACL. Ce exemple montre une ACL minimale. Si par exemple on ajoute permissions pour le utilisateur "jean" avec le commande \emph{setfacl} on va avoir les ACL étendu.

\begin{verbatim}
$ setfacl -m user:jean:rwx repertoire
$ getfacl --omit-header repertoire 
	user::rwx 
	user:jean:rwx	
	group::r-x 
	mask::rwx 
	other::---
\end{verbatim}

Il faut rappeler que le command \emph{setfacl} a été employer avec le modificateur \emph{-m (modify)}. Pour regarder les résultat on utilisé une autre fois le \emph{getfacl}. Il faut savoir que le modificateur \emph{--omit-header} cache les première 3 ligne avec les information sur les propriétaires.  

Tout d'abord on peut voir que l'entrée masque a été ajouter avec l'entrée d'utilisateur jean. Ces permission sont crée comme la union entré les permission de jean et du groupe propriétaire. La masque doit être une valeur que ne masque aucune permission, alors, ce valeur se trouve comme l'union de les élément que définissions la classe groupe.  

\begin{verbatim}
ls -dl repertoire
	drwxrwx---+ ... user group ... repertoire
\end{verbatim}

Si on rappel le dernière exemple, le groupe propriétaire n'y a pas le droit de écriture (\emph{group::r-x}), par contre, les groupe classe vient avec ce droit. C'est pour ça que quand on calcule effectivement le droit pour le groupe propriétaire dans le modèle acl, ce droit est toujours le union avec la masque.

Le prochaine exemple montre comme lest ACL sont modifiée pour l'emploie du command \emph{chmod} ou \emph{setfacl}. On va effacer le permission de écrit de la classe groupe. On doit voir que si il n'y a pas de masque, le command doit changer directement les permission dé l'entrée ACL du groupe propriétaire. 


\begin{verbatim}
$ chmod g-w repertoire 
$ ls -dl repertoire 
	drwxr-x---+ ... user group ... repertoire 
$ getfacl --omit-header repertoire 
user::rwx 
user:jean:rwx 	#effective:r-x
group::r-x 	
mask::r-x 
other::---

\end{verbatim}

Si une ACL entrée contient permission désactivé pour le masque, getfacl doit ajouter une commentaire qui montre ça différence. Il faut voir aussi qu'est-ce que se passe quand on rajoute le permission. 

\begin{verbatim}
$ chmod g+w repertoire 
$ ls -dl repertoire 
	drwxrwx---+ ... user group ... repertoire 
$ getfacl --omit-header repertoir 
user::rwx 
user:jean:rwx 
group::r-x
mask::rwx
other::---
\end{verbatim}

Cette exemple montre que les changes avec le commande \emph{chmod} ne sont pas destructives. Les opérations sont complètement réversibles et les permission avant et après les deux opération d'aller et retour sont les mêmes, une caractéristique très important de les ACL POSIX. 

\subsection*{Exemple des ACL par défaut}

Avec le modifier -d, on peut ajouter les ACL par défaut d'un répertoire. 

\begin{verbatim}
$ setfacl -d -m group:admin:r-x repertoire 
$ getfacl --omit-header repertoire 
user::rwx
user:jean:rwx
group::r-x 
mask::rwx
other::---
default:user::rwx
default:group::r-x
default:group:admin:r-x
default:mask::r-x
default:other::---
\end{verbatim} 

Les ACL par défaut vient après les ACL d'accès et sont préfixe pour le chaîne de caractères "default::". Le courent ce que quand on ajoute une nouvelle règle seulement dans les ACL d'accès comme on a faire pour le groupe \emph{admin} aucune chose change avec les ACL par défaut. Par contre, il y a une extension sur linux que ajoute de manière automatique le règle dans les ACL par défaut. 

Il faut voir que il n y a pas aucun entrée pour jean dans les défaut ACL, alors, jean n'aura pas accès dans le nouvelle objet crée dans le répertoire (sauf si il est membre de un groupe qui a les permission). 

D'accord avec le mécanisme d'héritage qu'on a vu dans le page \ref{sec:heritage}, le sub-repertoire doit hériter les ACL de sont parent. Par défaut, le command \emph{mkdir} utilise une \emph{mode parameter} de 0777 pour ce appel de système. Observez:

\begin{verbatim}
$ mkdir repertoire/subrep 
$ getfacl --omit-header repertoire/subrep 
user::rwx 
group::r-x 
group:admin:r-x 
mask::r-x 
other::--- 
default:user::rwx 
default:group::r-x 
default:group:toolies:r-x 
default:mask::r-x 
default:other::---
\end{verbatim}

Pour les fichiers ç'arrive aussi. Le commande \emph{touch} utilise le \emph{mode value} 0666. Tous les permission que ne sont pas présent dans le \emph{mode parameter} sont écraser.

Enfin, aucune permission a été écraser dans le groupe classe, cependant, la masque a été appliqué. Ce politique s'assure que par exemple les application comme les compilateurs peuvent marcher bien avec les ACL, alors que même que ils ne supportent ce mécanisme, il peuvent créer les fichier avec les permission restreint et après les donner les permission d'exécution et quand même, les groupes et autres permission doivent marcher comme attendu. 


