\documentclass[draft]{article}

%Packages
\usepackage[francais]{babel}
\usepackage[utf8]{inputenc} 
\usepackage{algorithmic}
\usepackage{algorithm}


%Info
\title{\textbf{Controle de Accés e le POSIX Access Control Lists(ACL)} \\ NE410 - Administration de Systéme }
\author{Dan Pham et Fabrício Nascimento}
\date{October 2009}

\begin{document}

\maketitle
\newpage
%----------------------------------------------------------------------------
\section*{Introduction}

%Le probleme et la solution plus simple
Quand l'objective c'est controller l'accès sur les données dans une système de fichiers, il y a plusieurs formes de reglement. Par default, les systèmes POSIX (Portable Operatin System Interface)\cite{ieee1,ieee2} ont une mechanisme que permettre de associer chaque entité avec un ensemble de régle, lequel est composé por une sequence d'octet que exprime le droit du propriétaire, de son groupe et des autres utilisateurs. 

%les limitation de ce solution la
Ce mode tradicional est assez simple et capable de adresser les problemen plus frequent. Par contre, il pose des limitation aux administrateur de système, lequel fréquemment doivent employer quelques configuration non évidentes afin d'être capable de exprimer ces besoin. Par exemple les application comme le serveur FTP Proftp\cite{ftp} ont ces exclusive façon de résoudre ces problèmes de droits pour acceder les objets du système de fichier.

% La solution ACL
À couse de remédier ces limitation presenté les unix permettent l'employ de les ACL.   

% Le reference aux texte originale
Cette article presenté une exposition sur les ACL POSIX, ces mode de fonctionnement, ces clés de succes et désavantages. Le texte est fortement basé en l'article de Andreas Gruembacher\cite{aclsuse} dont a été dans l'equipe que as ajouté le support aux ACL dans le noyaux linux pour les système de fichier ext2 et ext3, lequel est le système de fichier plus utilisé dans les monde UNIX.

\section*{Le POSIX 1003.1}

%regarder bien le passer
Après savoir la necessite de regler sur le domaine de sécurise et non seulement les ACL, une groupe a était forme pendant la definition de la famille de patron POSIX 1003.1. Les premieres documents POSIX qui ont été consideré ces question étions les document 1003.1e (\emph{System Application Programming Interface}) et 1003.2c (\emph{Shell and Utilities}), cependant, la primiere approximation de ce sujet était trop ambitieuse. Les groupe responsable pour le patronisation avait centre ces effort dans une tas assez grande de choses, lesquelles  comprenant \emph{Access Control Lists} (ACL), \emph{Audit}, \emph{Capability},\emph{ Mandatory Access Control }(MAC), et \emph{Information Labeling}\cite{aclsuse}.

En Janvier de 1998\cite{aclsuse} le financeament était fini, par contre, le travaille n'était pas prés. De toute façon le diseptieme brouillon a été publique quand meme\cite{posix17}.  

%the drafts that resulted in ACL
Les base des ACL sont lancé sur le systeme tradicional present usualement dans tous les systeme UNIX, allors, avant de preciser sur les ACL on parlerais du modele tradicional.

\subsection*{Système Tradicional}

%Les groups e les permission
Le modele traditional POSIX offre trois group de utilisateur qui sont le propriétaire, le group e les autres. Chaque group a une octet que indique les permission de lecture (\textbf{r}ead), écrire (\textbf{w}rite) et execution (e\textbf{x}ecute). La premiere classe fournit les permission pour le utilisateur que rempli le role de propriétaire, ensuite, vien les droits pour le groupe principal du propriétaire enfim les droites pour touts les autres utilisateurs. 
 
%Explication simple
Après les trois octets peut venir le \emph{Set User Id}, \emph{Set Group Id} et \emph{Sticky bit}, lequelles sont utilisé dans certain cases. Il faut faire attention avec le \emph{Sticky Bit}, il permit les utilisateur normale d'executer les utilitaire comment le administrateur(\emph{root}), par contre, quelque manque de sècurite peut compromettre le système entiere.

%Le droit du root
Seulement le \emph{root} peut créer les groups e changer les association de groupes. Celui-lá que aussi peut changer les propriétaire.  

\subsection*{Les ACL}

%L'histoire


%Qu'es-ce qu'est?
%Comment ça marche

En Janvier de 1998\cite{aclsuse}

Pour remédier à ce limitation "trusted" UNIX systems comme Trusted Solaris, Trusted Irix, Trusted AIX ont été developer avec 

Dans une modèle de de sécurise ACL, si quelque agent faire une requête pour acceder aux donnés, il faut consulte les ACL pour une entrée que permetre l'operation demandé.    



%changer le titre
\begin{algorithm}
\caption{Verifie se une utilisateur peut ou ne peut pas acceder une objet du système de fichier}
\label{algacl}
\begin{algorithmic}
\IF{the user ID of the process is the owner}
	\STATE the owner entry determines access
\ELSIF{the user ID of the process matches the qualifier in one of the named user entries}
	\STATE this entry determines access 
\ELSIF{one of the group IDs of the process matches the owning group and the owning group entry contains the requested permissions} 
	\STATE this entry determines access
\ELSIF{one of the group IDs of the process matches the qualifier of one of the named group entries and this entry contains the requested permissions}
	\STATE this entry determines access
 
\ELSIF{one of the group IDs of the process matches the owning group or any of the named group entries, but neither the owning group entry nor any of the matching named group entries contains the requested permissions} 
\STATE this determines that access is denied

\ELSE
	\STATE the other entry determines access.
\ENDIF
 

\IF{the matching entry resulting from this selection is the owner or other entry and it contains the requested permissions}
	\STATE access is granted 
\ELSIF{the matching entry is a named user, owning group, or named group entry and this entry contains the requested permissions and the mask entry also contains the requested permissions (or there is no mask entry)}
	\STATE access is granted
 
\ELSE
	\STATE access is denied.
\ENDIF
\end{algorithmic}
\end{algorithm}


\section*{Linux}

%A voir maintenant
\subsection*{ACL Implementation in Linux}

%noyaux patches

% getfacl

%setfacl

%systemes de fichier

% Patches that implement POSIX 1003.1e draft 17 ACLs have been available for various versions of Linux for several years now. They were added to version 2.5.46 of the Linux kernel in November 2002. Current Linux distributions are still based on the 2.4.x stable kernels series. SuSE and the United Linux consortium have integrated the 2.4 kernel ACL patches earlier than others, so their current products offer the most complete ACL support available for Linux to date. Other vendors apparently are still reluctant to make that important change, but experimental versions are expected to be available later this year.
% 
% The Linux getfacl and setfacl command line utilities do not strictly follow POSIX 1003.2c draft 17, which shows mostly in the way they handle default ACLs. See section 6.
% 
% At the time of this writing, ACL support on Linux is available for the Ext2, Ext3, IBM JFS, ReiserFS, and SGI XFS file systems. Solaris-compatible ACL support for NFS version 3 exists since March 3, 2003.

%----------------------------------------------------------------------------

\section*{Conclusion}

\begin{thebibliography}{9}

%\bibitem{ref}
%  Author,
%  \emph{title}.
%  Aditional

 
\bibitem{aclsuse}
  Andreas Gruenbacher,
  \emph{POSIX Acess Control Lists on Linux}.
  http://www.suse.de/~agruen/acl/linux-acls/online/,
  2003.

\bibitem{ieee1}
    IEEE Std 1003.1-2001 (Open Group Technical Standard, Issue 6), 
	Standard for Information Technology--Portable Operating System Interface (POSIX) 2001. 
	ISBN 0-7381-3010-9. 
	http://www.ieee.org/

\bibitem{ieee2}
    IEEE 1003.1e and 1003.2c: Draft Standard for Information Technology--Portable Operating System Interface (POSIX)--Part 1: System Application Program Interface (API) and Part 2: Shell and Utilities, draft 17 (withdrawn). 
	October 1997. 
	http://wt.xpilot.org/publications/posix.1e/

\bibitem{ftp}
	Mark Lowes: 
	Proftpd: 
	A User's Guide March 31, 2003. 
	http://proftpd.linux.co.uk/

\bibitem{posix17}
    Winfried Trümper: Summary about Posix.1e. Publicly available copies of POSIX 1003.1e/1003.2c. February 28, 1999. http://wt.xpilot.org/publications/posix.1e/

\end{thebibliography}


%http://www.suse.de/~agruen/acl/linux-acls/online/

\end{document}